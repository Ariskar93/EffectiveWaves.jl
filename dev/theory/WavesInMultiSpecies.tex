\pdfoutput=1

% !TEX TS?program = pdflatexmk
\documentclass[12pt, a4paper]{article}
% Use the option doublespacing or reviewcopy to obtain double line spacing
\usepackage[utf8]{inputenc}

\usepackage[style= bwl-FU, backend = biber, maxbibnames=99, maxcitenames=2,uniquelist=minyear, firstinits=true]{biblatex}

% \AtEveryBibitem{\clearfield{number}}
% \AtEveryBibitem{\clearfield{doi}}
\AtEveryBibitem{\clearfield{note}}
\AtEveryBibitem{\clearfield{urldate}}
\AtEveryBibitem{\clearfield{issn}}
\AtEveryBibitem{\clearfield{isbn}}
%
\addbibresource{Library.bib}

\usepackage{authblk}
\usepackage{color,graphicx,tikz}
\usetikzlibrary{positioning,arrows}
% The amssymb package provides various useful mathematical symbols
\usepackage{mathtools,amssymb,amsmath,mathdots,amsfonts}
\usepackage[mathscr]{eucal} %just for the font \mathscr
\usepackage{enumerate}
%\usepackage{amsthm}
%\usepackage{mathaccents}
\usepackage{setspace}
\newtheorem{remark}{Remark} [section]
\usepackage{hyperref}

% macros used across many documents for multiple scattering notation
%% Packages for commenting and striking through text %%%%
\usepackage[colorinlistoftodos,bordercolor=orange,backgroundcolor=orange!20,linecolor=orange,textsize=scriptsize]{todonotes}
\usepackage{soul}
\newcommand{\will}[1]{\todo[inline]{\textbf{Will: }#1}}
\newcommand{\art}[1]{\todo[inline]{\textbf{Artur: }#1}}
\newcommand{\david}[1]{\todo[inline]{\textbf{David: }#1}}
\newcommand{\mike}[1]{\todo[inline]{\textbf{Mike: }#1}}
\newcommand{\wrong}[1]{\textcolor{red}{\st{#1}}}
%%%--------------------------------------------------%%%

\newcommand \eff {*}
\newcommand \scatZ {Z}
\newcommand \scatZs {\zeta}
\newcommand \Q {\vec Q}
\newcommand \T {\vec T}
\newcommand \B {\mathcal B}
\newcommand \regS {\mathcal S}
\newcommand \M {\vec {\mathcal M}}
\newcommand \s {\mathbf s}
\newcommand \Ab {\mathcal A}
\newcommand \A [1] {\Ab_#1}
\newcommand \p {p}
\newcommand \prob {P}
\newcommand {\type} [1]{{\{#1\}}}
\newcommand \reg {\mathcal R_N}
\newcommand \reginf {\mathcal R_\infty}
\newcommand {\nfrac}[1] {\mathfrak n_{#1}}
\newcommand {\Lamo}{\vec \Lambda}
\newcommand {\Lam}[1]{\Lamo_{#1}}
\newcommand \ef{\mathrm{E}}
\newcommand \reflect{\mathrm{ref}}
\newcommand \inc{\mathrm{in}}
\newcommand \In{\mathrm{I}}
\newcommand \cs { S}
\newcommand \cl { L}
\newcommand \Out{\mathrm{o}}

% \newcommand \nfrac {\mathfrak n_0}
\newcommand{\ensem}[1]{\langle #1 \rangle}

\def\bga#1\ega{\begin{gather}#1\end{gather}} % suggested in technote.tex
\def\bgas#1\egas{\begin{gather*}#1\end{gather*}}

\def\bal#1\eal{\begin{align}#1\end{align}} % suggested in technote.tex
\def\bals#1\eals{\begin{align*}#1\end{align*}}

\renewcommand{\vec}[1]{\boldsymbol{#1}}
\renewcommand{\thefootnote}{\fnsymbol{footnote}}

\newcommand{\initial}[1]{{#1}_\circ}
\newcommand{\ii}{\textrm{i}}
\newcommand{\ee}{\textrm{e}}
\newcommand{\deriv}[2]{\frac{\partial{#1}}{\partial{#2}}}
\newcommand{\derivtwo}[2]{\frac{\partial^{2}{#1}}{\partial{#2}^2}}
\newcommand{\derivtwomix}[3]{\frac{\partial^{2}{#1}}{\partial{#2}\partial{#3}}}

\newcommand{\bb}{\mathcal B}
\newcommand{\R}{\mathbb{R}}
\newcommand{\filler}{\hspace*{\fill}}

\newcommand{\lit}{\hspace{0.2cm}}

\newtheorem{theorem}{Theorem}
\def \proof{\noindent {\bf \emph{Proof:}} }


% \DeclareMathOperator{\sign}{sgn}
% \DeclareMathOperator{\divergence}{div}
% \DeclareMathOperator{\tr}{tr}
% \DeclareMathOperator{\Ord}{\mathcal{O}}
% \DeclareMathOperator{\GRAD}{grad}
% \DeclareMathOperator{\DIV}{DIV}
% \DeclarePairedDelimiter{\ceil}{\lceil}{\rceil}


%%% Local Variables:
%%% mode: latex
%%% TeX-master: t
%%% End:


% \graphicspath{{../images/}}
%\graphicspath{{Media/}}

\doublespacing
% \setlength{\topmargin}{0cm} \addtolength{\textheight}{2cm}
\evensidemargin=0cm \oddsidemargin=0cm \setlength{\textwidth}{16cm}

\begin{document}

\title{Notes on effective waves in a multi-species material}

% \author{
% Artur L. Gower$^{a}$, Michael J. A. Smith$^{a}$, \\ William Parnell$^{a}$ and David Abrahams$^{a,b}$ \\[12pt]
% \footnotesize{$^{a}$ School of Mathematics, University of Manchester, Oxford Road, Manchester, M13 9PL, UK}\\
% \footnotesize{$^{b}$ Isaac Newton Institute for Mathematical Sciences, 20 Clarkson Rd, Cambridge CB3 0EH, UK}
% }

\author[$\dagger$]{Artur L.\ Gower}

\affil[$\dagger$]{School of Mathematics, University of Manchester, Oxford Road, Manchester M13 9PL, UK}

\date{\today}
\maketitle

\begin{abstract}
This Supplementary Material is a self-contained document providing further detail on the  calculation of effective wavenumbers for uniformly distributed multi-species inclusions. The formulae for multi-species cylinders and spheres are given here, in addition to expressions describing reflection from a halfspace filled with cylinders. Code to implement the formulas is given in \url{github.com/arturgower/EffectiveWaves.jl}.  For detailed derivations see \href{https://arxiv.org/abs/1712.05427}{our paper}~\parencite{gower_reflection_2017}, which shows how to introduce a pair-correlation between the species.
\end{abstract}

\noindent
{\textit{Keywords:} polydisperse, multiple scattering, multi-species, effective waves, quasicrystalline approximation, statistical methods }


\section{Effective waves for uniformly distributed species}
\label{sec:results}


We consider a halfspace $x>0$ filled with $S$ types of inclusions (species) that are uniformly distributed. The fields are governed by the scalar wave equation:
\begin{align}
  &\nabla^2 u + k^2 u = 0, \quad \text{(in the background material)} \\
  &\nabla^2 u + k^2_j u = 0, \quad \text{(inside the $j$-th scatterer)},
\end{align}
 The background and species material properties are summarised in Table~\ref{tab:properties}.
The goal is to find an effective homogeneous medium with wavenumber $k_*$, where waves propagate, in an
\href{https://en.wikipedia.org/wiki/Ensemble_average_(statistical_mechanics)}{ensemble average sense}, with the same speed and attenuation as they would in a material filled with scatterers. See~\cite{gower_effectivewaves.jl:_2017} for the code that implements the formulas below.

Below we present the effective wavenumber, for any incident wavenumber and moderate number fraction, when the species are either all cylinders or spheres\footnote{In principal these formulas can be extended to include different shaped scatterers by using Waterman's T-matrix \cite{waterman_symmetry_1971,varadan_multiple_1978,mishchenko_t-matrix_1996}}.
  For cylindrical inclusions we also present the reflection of a plane wave from this multi-species material.


\begin{table}[h]
\centering
\begin{tabular}{|l| l|}
  \hline
Background properties: &  wavenumber $k$ \hspace{0.8cm} density $\rho$  \hspace{0.25cm} sound speed $c$
\\ \hline
Specie properties: & number density $\nfrac j$ \hspace{0.1cm} density $\rho_j $
\hspace{0.1cm} sound speed $c_j$  \hspace{0.1cm} radius $a_j$
\\\hline
\multicolumn{2}{|l|}{
  total number density $\nfrac {}$   \hspace{0.2cm}  effective wavenumber $k_\eff$ \hspace{0.2cm} species min. distance $a_{j\ell} > a_j + a_\ell$
}
\\\hline
% 2D Incident wave & $u_\inc(x,y) = \ee^{\ii \mathbf k \cdot \mathbf x} \quad \text{with} \quad \mathbf k \cdot \mathbf x = k x \cos \theta_\inc  + k y \sin \theta_\inc $
\end{tabular}
\label{tab:properties}
\caption{Summary of material properties and notation. The index $j$ refers to properties of the $j$-th species. Note a typical choice for $a_{j\ell}$ is $a_{j\ell} = c (a_j + a_\ell)$, where $c=1.01$.}
\end{table}


% The scalar wave equation appears in electromagnetism when the magnetic field is polarised in the direction along the cylinders by replacing $c \mapsto c_0 / n$ and $\rho \mapsto \mu_\mathrm{r}$, where $c_0$ is the speed of light in vacuum, $n$ is the refractive index of the material,  and $\mu_\mathrm{r}$ is the linear magnetic susceptibility. And likewise, when the electric field is polarised along the cylinders, with the replacements $c \mapsto c_0 / n$   and $\rho \mapsto \varepsilon_{r}$, where $\varepsilon_\mathrm{r}$ is the linear electric susceptibility.

\section{Cylindrical species}

We consider an incident wave
\begin{gather}
  u_\inc =  \ee^{\ii \mathbf k \cdot \mathbf x} \quad \text{with} \quad \mathbf k \cdot \mathbf x = k x \cos \theta_\inc  + k y \sin \theta_\inc,
%   u_\inc(x,y) = \ee^{\ii \mathbf k \cdot \mathbf x}, \quad \text{with} \quad \mathbf k \cdot \mathbf x = \alpha  x  + \beta y,
% \notag \\
%   \text{where} \qquad  \alpha = k \cos \theta_\inc, \quad \beta = k \sin \theta_\inc,
  \label{eqns:incident}
\end{gather}
and angle of incidence $\theta_\inc$ from the $x$-axis, exciting a material occupying the halfspace $x>0$. Then, assuming low number density $\nfrac {}$ (or low volume fraction $\sum_\ell \pi a_\ell^2 \nfrac \ell$), the effective transmitted wavenumber $k_\eff$ becomes
\begin{align}
   k_\eff^2 = k^2 - 4 \ii \nfrac {} \ensem{f_\circ}(0) -  4 \ii \nfrac {}^2 \ensem{f_{\circ\circ}}(0)
   + \mathcal O(\nfrac {}^3),
  \label{eqn:SmallNfracDiscrete}
\end{align}
with $\ensem{f_\circ}$ and $\ensem{f_{\circ\circ}}$ given by \eqref{eqn:fcircs}. The above reduces to \cite{linton_multiple_2005} equation (81) for a single species in the low frequency limit; This equation (81) has been confirmed by several independent methods~\cite{martin_estimating_2010,martin_multiple_2008,chekroun_time-domain_2012,kim_models_2010}.

The ensemble-average reflected wave measured at $x <0$ is given by
\begin{equation}
   \langle u_\mathrm{ref}\rangle =
     \frac{\nfrac {}}{\alpha^2}\left [
 R_1 + \nfrac {} R_2
  \right]\ee^{-\ii \alpha x + \ii \beta y} + \mathcal O (\nfrac {}^3),
  \label{eqn:reflection_results}
\end{equation}
where
\begin{align}
  &R_1 = \ii \ensem{f_\circ}(\theta_\reflect), \quad \theta_\reflect = \pi - 2 \theta_\inc,
  \\
  &R_2 = \frac{2 \ensem{f_\circ}(0)}{k^2 \cos^2 \theta_\inc}\left [\sin \theta_\inc \cos \theta_\inc \ensem{f_\circ}'(\theta_\reflect)  - \ensem{f_\circ}(\theta_\reflect) \right] + \ii \ensem{f_{\circ\circ}}(\theta_\reflect),
\end{align}
which reduces to \cite{martin_multiple_2011} equations (40-41) for a single species, which they show agrees with other known results for small $k$.

The ensemble-average far-field pattern and multiple-scattering pattern are\footnote{Note we introduced the terminology ``multiple-scattering pattern''.}
\begin{align}
  &\ensem{f_\circ}(\theta) = - \sum_{\ell=1}^S  \sum_{n=-\infty}^\infty  \frac{\nfrac \ell}{\nfrac {}} \scatZ_\ell^n \ee^{\ii n \theta},
\notag \\
  &\ensem{f_{\circ\circ}}(\theta) = - \pi  \sum_{\ell,j=1}^S \sum_{m,n=-\infty}^\infty  a_{\ell j}^2 d_{n-m}(k a_{\ell j})  \frac{\nfrac \ell \nfrac j}{\nfrac {}^2}\scatZ_\ell^n \scatZ_j^m \ee^{\ii n \theta},
  \label{eqn:fcircs}
\end{align}
where $d_m(x) = J_m'(x) H_m'(x) + (1 - (m/x)^2)J_m(x) H_m(x)$, the $J_m$ are Bessel functions, the $H_m$ are Hankel functions of the first kind and $a_{\ell j} > a_\ell + a_j$ is some fixed distance. The  $\scatZ^m_j$ describe the type of scatterer:
\be
\scatZ^m_j = \frac{q_j J_m' (k a_j) J_m ( k_j a_j) - J_m (k a_j) J_m' (k_j a_j) }{q_j H_m '(k a_j) J_m(k_j a_j) - H_m(k a_j) J_m '(k_j a_j)} = \scatZ^{-m}_j,
\label{eqn:Zm}
\en
with $q = (\rho_j c_j)/(\rho c)$. For instance, taking the limits $q \to 0$ or $q \to \infty$, recovers Dirichlet or Neumann boundary conditions, respectively.

\subsection{Any volume fraction}

The series expansions for low number density (or volume fraction) do not work when the particles are strong scatterers. In these cases we need to use formulas valid for any volume fraction.

Borrowing equations (45 - 47) from~\cite{gower_reflection_2017} we have
\begin{align}
    &k_\eff \sin \theta_\eff = k \sin \theta_\inc  \quad \text{with} \quad \mathbf k_\eff =(\alpha_\eff, \beta) := k_\eff(\cos\theta_\eff, \sin\theta_\eff),
    \label{eqn:Snells}
  \\
    &  \sum_\ell \sum_{n=-\infty}^\infty ( 2 \pi\nfrac \ell
  \mathcal Q^{n-m}_{j \ell}(k_\eff) \scatZ^n_\ell + \delta_{m n} \delta_{j \ell}  ) \A n_\ell
   = 0,
 \label{eqn:AmT}
\\
  &
   2\sum_{n=-\infty}^\infty \ee^{\ii n (\theta_\inc - \theta_\eff)} \sum_\ell
   \nfrac \ell \scatZ^n_\ell \A n_\ell = (\alpha_\eff-\alpha) \ii \alpha,
 \label{eqn:AmInc}
 \end{align}
in terms of the unknown parameters $\A n_\ell$ and $k_\eff$, where
\begin{align}
  &\mathcal Q^n_{j \ell}(k_\eff) =  \frac{\mathcal N_{n}(ka_{j\ell},k_\eff a_{j\ell})}{k^2 - k_\eff^2}  +  \mathcal X_{n}(\s_j,\s_\ell),
   \\
  & \mathcal N_n(x,y) = x H_n'(x) J_n(y) - y H_n(x) J_n'(y),
\label{eqn:Nn}
\end{align}
and $\mathcal X_{n} = 0$ for {\it hole correction}, or for a more general pair distribution
\begin{equation}
   \mathcal X_{n}(\s_j,\s_\ell) =  \int_{a_{j \ell}< R < \bar a_{j \ell}} H_{n}(k R) J_{n}(k_\eff R) \chi(R | \s_j, \s_\ell)  R \, d R,
   \label{eqn:X_eff}
\end{equation}
where we assume that when the distance between two cylinders $R_{j \ell} > \bar a_{j \ell}$, then the pair correlation is the same as hole correction.

In the notation given in~\cite{gower_reflection_2017} we replaced $\A m_\eff(\s_2) \to \A m_\ell$, $\p(\s_2) \to \delta(\s_2 - \s_\ell) \frac{\nfrac \ell}{\nfrac {}}$,  $\nfrac {} = \sum_\ell \nfrac j$, $\nfrac {} = \sum_\ell \nfrac j$,
$\scatZ^n(\s_2) \to \scatZ^n_j$, $\mathcal X_{\eff} \to \mathcal X_{n-m}(\s_j,\s_\ell)$ and here we assumed no boundary-layer $\bar x = 0$.
%
% Rewriting in terms of infinite dimensional tensors
% \begin{align}
%    & (\vec A_\ell)_n = \A n_\ell, \quad  (\vec z_\ell)_n =  \sqrt{\scatZ^n_\ell},
%     % \;\; n=-N,\ldots, N,
%    \\
%    & (\vec { Q}_{j \ell})_{mn} =  \mathcal Q^{n-m}_{j \ell}(k_\eff),
%  \end{align}
%
% \begin{align}
%     &  \sum_\ell \sum_{n=-\infty}^\infty ( 2 \pi\nfrac \ell
%   \vec {Q}_{j \ell}(k_\eff) \scatZ^n_\ell + \delta_{m n} \delta_{j \ell}  ) \A n_\ell
%    = 0,
% \\
%   &
%    2\sum_{n=-\infty}^\infty \ee^{\ii n (\theta_\inc - \theta_\eff)} \sum_\ell
%    \nfrac \ell \scatZ^n_\ell \A n_\ell = (\alpha_\eff-\alpha) \ii \alpha,
%  \end{align}

Now we approximate (\ref{eqn:AmInc},\ref{eqn:AmT}) by summing $n$ from $-N$ to $N$ and then rewriting these equations as
\begin{align}
    &  \sum_\ell \vec M_{j \ell} \vec A_\ell
   = 0, \implies \det (\vec M_{j \ell}) = 0,
   \label{eqn:secular}
% \\
%   &
%     \sum_\ell \vec Z_\ell \cdot \vec A_\ell = (\alpha_\eff-\alpha) \ii \alpha,
%    \label{eqn:effective_amplitude}
 \end{align}
where
\begin{align}
   & (\vec A_\ell)_n = \A n_\ell, \quad
  %  (\vec Z_\ell)_n =  \scatZ^n_\ell,
    % \;\; n=-N,\ldots, N,
   (\vec M_{j \ell})_{mn} =  2 \pi\nfrac \ell \scatZ^n_\ell \mathcal Q^{n-m}_{j \ell}(k_\eff) + \delta_{m n} \delta_{j \ell},
 \end{align}
and $n, m = -N, -N +1, \ldots N$.

The strategy to solve these equations is to: find $k_\eff$ such that the determnant in~\eqref{eqn:secular} is zero and then find the eigenvector $\mathbb A$ of $(\vec M_{j \ell})$ with the smallest eignvalue; use Snell's law~\eqref{eqn:Snells} to find $\theta_\eff$; finally use~\eqref{eqn:AmInc} to determine the magnitude of $\mathbb A$.

One concern, is that the solutions $k_\eff$ to \eqref{eqn:secular} are not unique.

\subsubsection{Reflection coefficient}

Borrowing equation (88) from~\cite{gower_reflection_2017}, the average reflection from a halfspace is
\begin{align}
  &\ensem{u_\mathrm{ref}(x,y)} = \ee^{\ii k( - x \cos \theta_\inc + y \sin \theta_\inc)} R,
  \\
  & R =  \frac{2 \ii }{\alpha (\alpha + \alpha_\eff)} \sum_{\ell = 1}^S \sum_{n=-\infty}^\infty \ee^{\ii n \theta_\mathrm{ref}} \nfrac \ell \A n_\ell \scatZ^n_\ell,
\end{align}
with $\theta_\mathrm{ref} = \pi - \theta_\inc - \theta_\eff$.


\section{Spherical species}
\label{sec:Effective3D}

The results here are derived by applying the theory in our paper to the results in \cite{linton_multiple_2006}. We omit the details as the result follows by direct analogy.

For spherical inclusions the transmitted wavenumber becomes,
\begin{align}
   k_\eff^2 = k^2 -\nfrac {} \frac{4 \pi \ii}{k}  \ensem{F_\circ}(0) + \nfrac {}^2 \frac{(4 \pi)^2 }{k^4} \ensem{F_{\circ\circ}}
   + \mathcal O(\nfrac {}^3),
  \label{eqn:SmallNfracSpheres}
\end{align}
where for spheres we define the ensemble-average far-field pattern and multiple-scattering pattern,
\begin{align}
  &\ensem{F_\circ}(\theta) = - \sum_{n=0}^\infty P_n (\cos \theta ) \sum_{j=1}^S (2n+1)  \scatZs^n_j \frac{\nfrac j}{\nfrac{}},
\\
  &\ensem{F_{\circ\circ}} = \frac{\ii (4 \pi)^2}{2} \sum_{n,p=0}^\infty \sum_{j,\ell=1}^S \sum_q  \frac{\sqrt{(2n+1)(2p+1)}}{(4 \pi)^{3/2}}  \sqrt{2 q +1} \mathcal G(n,p,q) k a_{j\ell} D_q(k a_{j\ell}) \scatZs^n_j \scatZs^p_\ell \frac{\nfrac j \nfrac \ell}{\nfrac{}^2},
\notag
\end{align}
where
\[
D_m(x) = x j_m'(x)( x h_m'(x) + h_m(x)) + (x^2 - m(m+1))j_m(x) j_m(x),
\]
$P_n$ are   Legendre polynomials, $j_m$ are   spherical Bessel functions, $h_m$ are spherical Hankel functions of the first kind  and
\be
\scatZs^m_j = \frac{q_j j_m' (k a_j) j_m ( k_j a_j) - j_m (k a_j) j_m' (k_j a_j) }{q_j h_m'(k a_j) j_m(k_j a_j) - h_m(k a_j) j_m '(k_j a_j)} = \scatZs^{-m}_j,
\label{eqn:Zm}
\en
with $q = (\rho_j c_j)/(\rho c)$, where the $\mathcal G$ is a Gaunt coefficient and is equal to
\[
\mathcal G(n,p,q) = \frac{\sqrt{(2n+1)(2p+1)(2q+1)}}{2 \sqrt{4 \pi}}   \int_0^\pi  P_n(\cos \theta) P_p(\cos \theta) P_q(\cos \theta) \sin \theta d \theta.
\]
See \cite{caleap_effective_2012} for details on reflection from a single species, although, to our knowledge, a formula for reflection from a single species valid for moderate number fraction and any wavenumber has not yet been deduced.

\printbibliography

% \bibliographystyle{RS}
% \bibliography{../Library2}

\end{document}
